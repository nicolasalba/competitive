\section*{Erdős–Szekeres Theorem}
This theorem is related to increasing and decreasing sequences.

Suppose \( a, b \in \mathbb{N} \), \( n = ab + 1 \), and \( x_1, x_2, \dots, x_n \) is a sequence of \( n \) real numbers. Then this sequence contains a monotonic increasing (decreasing) subsequence of \( a + 1 \) terms or a monotonic decreasing (increasing) subsequence of \( b + 1 \) terms. Dilworth's lemma is a generalization of this theorem.

\section*{Grundy Numbers in Game Theory}
Grundy numbers are used in game theory to analyze games that can be represented as directed state graphs. In these graphs, if a player loses in a state, its Grundy number is zero; otherwise, it is a positive number. The Grundy number for each vertex is defined as:

\[
\text{Grundy}(\text{losing state with no moves}) = 0
\]
\[
\text{Grundy}(\text{vertex}) = \text{MEX}(\text{adjacent\_nodes}[\text{vertex}])
\]

where MEX stands for the "minimum excludant," which is the smallest non-negative integer not present in the set of Grundy numbers of adjacent nodes.

If you have multiple independent games, the final Grundy number is calculated as:

\[
\text{Grundy}(\text{game}_1) \oplus \text{Grundy}(\text{game}_2) \oplus \text{Grundy}(\text{game}_3) \oplus \dots \oplus \text{Grundy}(\text{game}_n)
\]

where \( \oplus \) denotes the bitwise XOR operation.
